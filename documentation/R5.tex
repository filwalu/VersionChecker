\chapter{Podsumowanie}
W ramach projektu Aplikacja pozwalająca na szybkie sprawdzenie wersji serwisów na serwerach Linux szereg prac, w tym:
\begin{itemize}
    \item Implementacja systemu logowania błędów do pliku app.log za pomocą biblioteki loguru.
    \item Uruchomienie prostego serwera www działającego na lokalnym hoście, dzięki bibliotece Flask.
    \item Przygotowanie interfejsu użytkownika na stronie www za pomocą templateów Jinja, kodu CSS oraz JavaScript.
    \item Wykorzystanie plików JSON do przechowywania danych o serwisach i wersjach, oraz docelowych hostach.
    \item Wykorzystanie biblioteki paramiko do obsługi połączeń SSH do docelowych hostów, za pomocą prywatnego klucza.
\end{itemize}


W dalszej kolejności planowane są następujące prace rozwojowe:
\begin{itemize}
    \item Konteneryzacja aplikacji z wykorzystaniem Docker, co ułatwi wdrożenie i skalowalność.
    \item Rozbudowa obsługi wyjątków dla zwiększenia stabilności i niezawodności aplikacji.
    \item Wdrożenie obsługi requestów /POST w celu możliwości dodania pliku hosta bez potrzeby umieszczania go bezpośrednio w plikach.
\end{itemize}
