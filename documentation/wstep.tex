\chapter*{Wstęp}

W dzisiejszych czasach zarządzanie wersjami serwisów na serwerach Linux staje się coraz bardziej skomplikowane z powodu rosnącej liczby serwisów i ich zależności. Codzienne wyzwania związane z utrzymaniem aktualności i bezpieczeństwa systemów wymagają innowacyjnych i efektywnych rozwiązań. Nasz projekt, wykorzystujący zaawansowane technologie Python i framework Flask, ma na celu wprowadzenie kompleksowego systemu do sprawdzania wersji serwisów, który zarówno usprawni zarządzanie serwerami, jak i znacząco podniesie komfort pracy administratorów systemów.

Dążymy do rozwiązania problemów związanych z monitorowaniem wersji serwisów przez automatyzację procesów, stosując nowoczesne biblioteki takie jak Paramiko do bezpiecznego połączenia z serwerami oraz interaktywne rozwiązania frontendowe. Nasze podejście ma na celu nie tylko ułatwienie zarządzania wersjami oprogramowania, ale także zapewnienie bezpieczeństwa i stabilności systemów poprzez szybkie i efektywne aktualizowanie informacji o serwisach. W ten sposób przyczyniamy się do poprawy ogólnej jakości zarządzania infrastrukturą IT, łącząc praktyczne umiejętności programistyczne z realnymi wyzwaniami administracji serwerami.

Poprzez ten projekt chcemy zademonstrować, jak nowoczesne technologie mogą być wykorzystane do automatyzacji i optymalizacji procesów zarządzania systemami informatycznymi, co w efekcie przyniesie korzyści zarówno dla administratorów, jak i użytkowników końcowych.