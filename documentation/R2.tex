\chapter{Opis struktury projektu}
\section{Komponenty i Organizacja Aplikacji}

Projekt Systemu Sprawdzania Wersji Serwisów na Serwerach Linux składa się z kilku głównych komponentów:

\begin{itemize}
\item \textbf{Backend}: Flask, Python
\item \textbf{Frontend}: HTML, CSS, JavaScript
\item \textbf{Skomunikowanie z serwerami Linux}: Paramiko do obsługi SSH
\item \textbf{Repozytorium kodu i kontrola wersji}: Git, GitHub
\end{itemize}

Główne skrypty i pliki:

\begin{itemize}
\item \texttt{app.py}: Główny plik aplikacji Flask, zarządza routingiem i logiką backendu.
\item \texttt{handler/watcher.py}: Skrypt do monitorowania zmian w folderze hostów.
\item \texttt{templates/}: Katalog zawierający szablony HTML dla Flask.
\item \texttt{frontend/static/}: Katalog zawierający pliki statyczne takie jak CSS i JavaScript.
\item \texttt{handler/results/}: Katalog przechowujący wyniki wersji serwisów dla poszczególnych hostów.
\item \texttt{handler/fetch/hosts/}: Katalog przechowujący pliki JSON z danymi hostów.
\end{itemize}


Struktura projektu jest zaprojektowana w taki sposób, aby maksymalizować ponowne wykorzystanie kodu i ułatwić rozszerzanie systemu o nowe funkcjonalności.

\section{Opis Techniczny Projektu}

Projekt został zrealizowany w języku Python z wykorzystaniem frameworka Flask do obsługi backendu oraz HTML, CSS i JavaScript do frontendowej części aplikacji. Do zarządzania projektem i kodem źródłowym wykorzystano środowisko Visual Studio Code oraz system kontroli wersji Git.

\section{Instalacja i uruchomienie po stronie serwera}

Aby uruchomić aplikację na serwerze, należy wykonać następujące kroki:

\begin{enumerate}
    \item Sklonuj repozytorium z GitHub:
    \begin{verbatim}
    git clone https://github.com/filwalu/VersionChecker.git
    \end{verbatim}
    \item Przejdź na gałąź \texttt{main}:
    \begin{verbatim}
    git checkout main
    \end{verbatim}
    \item Przejdź do katalogu \texttt{app}:
    \begin{verbatim}
    cd app/
    \end{verbatim}
    \item Utwórz wirtualne środowisko:
    \begin{verbatim}
    python3 -m venv .venv
    \end{verbatim}
    \item Zainstaluj wymagane pakiety:
    \begin{verbatim}
    pip3 install -r requirements.txt
    \end{verbatim}
    \item Uruchom aplikację:
    \begin{verbatim}
    python3 app.py
    \end{verbatim}
\end{enumerate}

\section{Po stronie użytkownika}

Aby uzyskać dostęp do aplikacji, użytkownik powinien w przeglądarce internetowej wejść na adres:

\begin{verbatim}
127.0.0.1:5001
\end{verbatim}

\noindent W ten sposób użytkownik połączy się z uruchomioną aplikacją.


\subsection{Wymagania systemowe}

Aplikacja jest zaprojektowana z myślą o niskich wymaganiach sprzętowych:

\begin{itemize}
\item Procesor: 1 GHz lub szybszy.
\item Pamięć RAM: 512 MB.
\item Przestrzeń na dysku: 100 MB.
\item System operacyjny: Linux, Windows, MacOS.
\end{itemize}

\subsection{Mechanizm zarządzania danymi}

Projekt wykorzystuje pliki JSON do przechowywania danych o serwisach i wersjach, co pozwala na prostą i efektywną manipulację danymi bez potrzeby korzystania z zewnętrznych systemów DBMS. Struktura plików JSON jest zaprojektowana w taki sposób, aby umożliwić szybkie odczytywanie i zapisywanie stanu serwisów oraz informacji o wersjach, co zapewnia wysoką wydajność działania systemu.

\section{Repozytorium i System Kontroli Wersji}

Projekt wykorzystuje system kontroli wersji Git, co umożliwia skuteczne zarządzanie historią zmian kodu źródłowego. Repozytorium kodu znajduje się na platformie GitHub pod adresem:\ \url{https://github.com/filwalu/VersionChecker} i będzie dostępne publicznie do dnia 30.09.2024. Bezpieczne połączenie z repozytorium zabezpieczono za pomocą pary kluczy SSH. Poniżej przedstawiono opis użytych poleceń Git:

\begin{enumerate}
\item \texttt{git init} - inicjalizacja nowego repozytorium Git.
\item \texttt{git clone [URL]} - klonowanie repozytorium przy użyciu SSH.
\item \texttt{git add -A} - dodawanie zmian do kolejki commitów.
\item \texttt{git status} - sprawdzanie statusu zmian.
\item \texttt{git commit -m "[wiadomość]"} - commitowanie zmian z opisem.
\item \texttt{git push} - wysyłanie zmian do zdalnego repozytorium przez SSH.
\item \texttt{git merge [branch]} - scalanie zmian z wybranej gałęzi do bieżącej gałęzi.
\end{enumerate}

\clearpage